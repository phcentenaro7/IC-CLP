\section{Algorithms and programming models}

Commonplace in this report are discussions regarding algorithms and programming models. In this section, we explore useful terminology and mathematical concepts for later sections.

\subsection{Algorithmic terminology}

The \emph{Dictionary of Algorithms and Data Structures} \cite{BLACK2021} defines algorithm as ``A computable set of steps to achieve a desired result.`` Consequently, any computer program is an algorithm, since it requires the execution of a sequence of instructions to reach its goals.

Often, memory and runtime are the metrics used to evaluate the performance of an algorithm. In both cases, the problems used to benchmark performances are extreme -- after all, computers must usually solve problems with large inputs, and algorithms that can better deal with such problem sizes are preferable.

A mathematical approach to algorithmic performance revolves around their \emph{asymptotic} behavior; that is, ``As the problem size increases, how does memory and time consumption of the algorithm increase?``



\subsection{Programming models and methods}